\chapter{Fazit und Aussicht für den Prototypen}
Der Prototyp ist mit dem neuronalen Netz verbunden. Zur Erkennung und Verwaltung von Interaktionen, ist dies eine gute Möglichkeit, um Steuerungen wie Smartphones oder ähnliche Geräten zum Teil zu ersetzen. Da kein Einfluss auf die Automatisierung oder Steuerung einzelner Geräte genommen werden kann, sind Smartphones oder ähnliche Geräte eine optionale Steuerungsmöglichkeit für kleinere Veränderungen der smarten Geräte oder Möbelstücke. Dies hat den Nachteil für eingeschränkte Personen. Personen die körperlich oder geistig eingeschränkt sind, haben mit der Sofasteuerung den Vorteil, sich nicht um die smarten Geräte durch ein Smartphone oder ähnlichen Geräten kümmern zu müssen.
\newline
Dem Autor ist bei der Datensammlung mit den Sensoren besonders aufgefallen, dass das Sofa wie ein herkömmliches Sofa genutzt werden muss, damit die Nutzer am besten damit klar kommen. Wichtig ist, dass die Nutzer bei der Einrichtung eines solchen Systems das Sofa aktiv nutzen, bevor das neuronale Netz die Interaktionen verwaltet. So kann spezifisch darauf eingegangen werden, wie der Nutzer das Sofa im Alltag benutzt.
\newline
\newline
Der Autor möchte auch nochmal betonen, dass die Steuerung besonders für Personen geeignet ist, welche ihren Alltag nicht ohne Hilfe bewältigen können. Die Interaktionen mit dieser Steuerung führen zu einer passiven Automatisierung bzw. Steuerung smarter Endgeräte oder Möbelstücke, wodurch Nutzer sicht nicht mehr darum kümmern müssen.

\section{Zusammenfassung der Bachelorarbeit}
Mit dieser Bachelorarbeit will der Autor zeigen wie es möglich ist, die Innenausstattung in das Smart Home zu integrieren. Anhand eines Beispiels an einem Sofa zeigt der Autor dessen Umsetzung. Der Prototyp besteht aus mehreren ESP32 und einem Raspberry Pi 3 sowie verschiedenen Sensoren, die an den ESPs angeschlossen sind. Die Evaluation der empirischen Methode zeigt, dass der Ultrasonic-Sensor gegen FSR-Sensoren ausgetauscht werden muss. Da die Messung an den Sofalehnen dadurch beeinträchtigt wird. Dennoch lässt sich aus dem dritten Teil der Arbeit zur Architektur sagen, dass diese ohne Probleme als Prototyp umgesetzt wird.
\newline
\newline
Weiterhin sollen die Interaktionen mit diesem Prototypen verwaltet werden. Diese Aufgabe übernimmt ein neuronales Netz. Das neuronale Netz unterteilt die Ergebnisse am Ende in drei Klassen. Es lernt diese Klassifizierungen richtig vorhzusagen, indem zwei Datensätze basierend auf den gleichen Sensorwerten trainiert werden. Ein Datensatz behält alle Datenpunkte und ein zweiter Datensatz bekommt nur die gefilterten Interaktionen. Das Ergebnis zeigt deutlich, dass der Datensatz mit allen Datenpunkten eine Fehlerrate von 0,03 hat und damit wesentlich besser klassifiziert. Im Vergleich dazu hat der zweite Datensatz mit einer Fehlerrate 0,16 schlechter abgeschnitten. Also wird nochmal gezeigt, dass es sehr wichtig ist, dass die Sensorwerte so genau wie möglich sind. 
\newline
Für das neuronale Netz ist es auch sehr wichtig, wie es aufgebaut ist. Es macht einen Unterschied welche Aktivitätsfunktionen benutzt werden und ob der Bias-Wert hinzugezogen wird. Zusätzlich muss die Lernrate angepasst werden indem bei beim höchsten Wert angefangen wird und die Lernrate immer weiter runter geht. Die Normalisierung ist wichtig, wenn bei TensorFlow ein eigener Datensatz benutzt wird. TensorFlow bietet daneben auch schon vorbereitete, skalierte Datensätze an. 
\newline
\newline
Der Prototyp hat von der Sicht des Autors bei Weiterentwicklung gute Aussichten auf eine reale Einbindung in ein Smart Home. Damit zieht der Autor das Fazit, dass Möbel sich in ein Smart Home als smarte Erweiterung einbinden lassen. Wenn das Smart Home bestimmte Möbel, wie Sitzmöglichkeiten zur Steuerung nutzt, dann ist es außerdem ein Vorteil, wenn dafür Machine Learning Methoden eingesetzt werden. Dies zeigt die Verbindung des Sofas, mit der Verwaltung durch das neuronale Netz. Die Grundmerkmale aus dieser Arbeit sind also ein Prototyp zur Erfassung der Sensordaten aus den Interaktionen mit dem Sofa und ein neuronales Netz zur Verwaltung dieser Interaktionen. Somit ist es also immer wichtig in einer Steuerung, eine Messung von Daten zu implementieren und eine Methode diese zu verarbeiten. Somit hat durch die komplette Arbeit hinweg den Autor immer wieder beschäftigt, wie diese Aufgaben richtig aufgeteilt werden. 

\section{Zukunftsaussicht für das System}
Da der Prototyp noch nicht vollkommen für ein Smart Home bereit ist, gibt es für die Zukunft weitere Punkte die zur Verbesserung beitragen. Wichtig dabei ist, dass jede Komponente weiterhin nur für eine Aufgabe zuständig ist. Zudem wird des öfteren in der Arbeit erwähnt, dass der Ultrasonic-Sensor nicht die ideale Lösung ist. Daher ist es besser, wenn entweder nur FSR-Sensoren eingebaut sind oder Sensoren benutzt werden, die die komplette Fläche eines Sofakissens oder Armlehne einnehmen. Der Vorteil beim implementieren mehrerer Sensoren auf einer Fläche ist, die damit höhere Genauigkeit der Interaktionen. Um herauszufinden welche Sensoren weiterhin benutzt oder ersetzt werden, muss der Prototyp weiter von Probanden getestet werden. Um die Situationen bei diesen Tests realistischer zu gestalten, bietet die Implementierung von smarten Endgeräten in den Prototypen ein genaueres Umfeld in Smart Homes. Weiterhin sollten Probanden hinzugezogen werden, an die dieser Prototyp gerichtet ist. Im ersten Teil in Kapitel \ref{sec:persons} beschreibt der Autor diese Personen.
\newpage
Das neuronale Netz muss in Zukunft dann auch weiter angepasst werden. Da sich die Sensorwerte dann noch ändern und erweitern werden. So kann noch genauer klassifiziert werden. Dadurch werden außerdem die Möglichkeiten größer im Output-Layer mehr Ergebnisse zur Verwaltung zu benutzen. Also kann die Automatisierung und Steuerung der smarten Geräte individueller gestaltet werden. Für Personen die körperlich oder geistig eingeschränkt sind, ist die Selbstständigkeit und Sicherheit dadurch weiterhin gewährleistet und wird dann immer besser an die Person angepasst. In Zukunft ist es also weiterhin ein System, welches sich an die genannten Personen richtet.
\newline
\newline
Als abschließende Bemerkung vom Autor, zeigt die Bachelorarbeit, dass der Prototyp darstellt wie die Innenausstattung mit in ein Smart Home implementiert werden kann und dies hat besonders für eingeschränkte Personen die Hilfe benötigen einen Vorteil zur Selbstständigkeit und Sicherheit.