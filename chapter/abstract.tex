\chapter*{Vorwort zum Aufbau der Bachelorarbeit}
Die Bachelorarbeit ist sechs Themen unterteilt. Die Einleitung beschreibt, wie der Autor auf das Thema kommt und welche Personen für das System vorgesehen sind. Zusätzlich wird das Ziel der Bachelorarbeit erläutert und welche Relevanz das Thema Smart Homes hat.
\newline
\newline
Im weiteren Verlauf beschreibt das Thema Grundlagen, die für den weiteren Verlauf der Bachelorarbeit benötigt werden, um zu verstehen worum es in der Bachelorarbeit geht.
Der Hauptteil dieser Arbeit zeigt die theortische und praktische Umsetzung des neuronalen Netzes. Dort wird beschrieben, wie das neuronale Netz geplant ist und die Umsetzung in Python. Die Evaluation erläutert das Training des neuronalen Netzen durch verschiedene Probanden. Zusätzlich werden dort die Ergebnisse vorgestellt und wie sich das neuronale Netz durch das Training entwickelt hat. 
\newline
\newline
Zum Schluss wird ein Fazit und eine Zusammenfassung zum Thema Smart Home Interior geschlossen.

\newpage
\chapter*{Abstract}
Die Bachelorarbeit Smart Home Interior behandelt die Innenausstattung von Smart Homes. Als Beispiel wird ein Sofa zur Interaktion mit smarten Geräten verwendet. Ein neuronales Netz steuert durch Klassifizierung die Interaktionen. Es werden als Eingabewerte für das neuronale Netz Codes übergeben, die von den ESP32 mit dem MQTT Protokoll verschickt werden. Die Codes werden entsprechend verschickt, je nachdem welcher Sensor besetzt ist. Also ist das Ziel der Bachelorarbeit, das eine Person mit dem Sofa interagiert und das neuronale Netz die Interaktion auswerten kann, damit entsprechende smarte Geräte oder Möbel gesteuert werden. Das neuronale Netz ersetzt, dass zuvor entwickelte Regelsystem, welches statisch - Sitz- und Liegepositionen erkennt.

