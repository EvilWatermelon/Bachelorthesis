\chapter*{Vorwort zum Aufbau der Bachelorarbeit}
Die Arbeit gliedert sich in sechs Teile. Der erste Teil widmet sich der Einleitung und einem Einblick in den Inhalt. Daraufhin werden im zweiten Teil die Grundlagen behandelt. Diese sollen ein Verständnis vermitteln um die weiteren Kapitel der Bachelorarbeit zu verstehen. 
\newline
\newline
Anschließend behandelt der Autor im dritten Teil der Arbeit die Architektur des Prototyps, welcher in der Bachelorarbeit benutzt wird. Zusätzlich behandelt dieser Teil die Architektur des Prototypen aus dem Praxisprojekt um die späteren Veränderungen zu veranschaulichen. Dies ist notwendig, da der Autor die Veränderungen in den weiteren Teilen der Bachelorarbeit beschreibt.
\newline
Auf der Grundlage aus dem zweiten Teil der Arbeit wird die Umsetzung des neuronalen Netzes im vierten Teil beschrieben. Außerdem wird ein vergleich gezogen aus dem Regelsystem, welches im Rahmen des Praxisprojekts entwickelt wurde.
\newline
Weiterhin beschreibt das fünfte Kapitel die Evaluation aus der Sammlung der Trainingsdaten und der Ergebnisse aus dem neuronalen Netz.
\newline
\newline
Abschließend findet im sechsten Teil ein Fazit statt, welches die Endergebnisse der Arbeit veranschaulicht und wie der Prototyp in Zukunft weiterentwickelt werden kann.
Als Anhang findet man Teile aus dem Programmcode der Programme die für die Bachelorarbeit geschrieben wurden.

\newpage
\chapter*{Abstract}


