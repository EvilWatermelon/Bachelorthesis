\chapter*{Vorwort zum Aufbau der Bachelorarbeit}
Die Arbeit gliedert sich in sechs Teile. Der erste Teil widmet sich der Einleitung und einem Einblick in den Inhalt. Daraufhin werden im zweiten Teil die Grundlagen behandelt. Diese sollen ein Verständnis vermitteln, um die weiteren Kapitel der Bachelorarbeit zu verstehen. 
\newline
\newline
Anschließend behandelt der Autor im dritten Teil der Arbeit die Architektur des Prototypen, welcher in der Bachelorarbeit benutzt wird. Zusätzlich behandelt dieser Teil die Architektur des Prototypen aus dem Praxisprojekt um die späteren Veränderungen zu veranschaulichen. Dies ist notwendig, da der Autor die Veränderungen in den weiteren Teilen der Bachelorarbeit beschreibt.
\newline
Auf der Grundlage aus dem zweiten Teil der Arbeit wird die Umsetzung des neuronalen Netzes im vierten Teil beschrieben. Außerdem wird ein Vergleich aus dem Regelsystem gezogen, welches im Rahmen des Praxisprojekts entwickelt wurde.
\newline
Weiterhin beschreibt das fünfte Kapitel die Evaluation aus der Sammlung der Trainingsdaten und der Ergebnisse aus dem neuronalen Netz.
\newline
\newline
Abschließend findet im sechsten Teil ein Fazit statt, welches die Endergebnisse der Arbeit veranschaulicht und wie der Prototyp in Zukunft weiterentwickelt werden kann.
Als Anhang findet man Teile aus dem Programmcode der Programme, die für die Bachelorarbeit geschrieben wurden.
\newpage
\chapter*{Abstract}
Diese Bachelorarbeit behandelt ein System bestehend aus Sensoren, welche mit einem ESP32 und Raspberry Pi einen Sofaprototypen bilden. Der Prototyp zur Sammlung der Sensorwerte und ein neuronales Netz sind die beiden Kernkomponenten. Der aktuelle Prototyp hat Ultrasonic-Sensoren angeschlossen, welche zur Messung der Rückenlehne benutzt werden. Dieser Sensor ist nicht geeignet, da dieser zu auffällig ist und nur im Radius vor der Rückenlehne messen kann. Daher sollte dieser durch einen FSR-Sensor ersetzt werden. Das neuronale Netz kann noch nicht zu 100\% Vorhersagen treffen, kommt mit einer Vorhersage von 99,7\% aber schon nah dran. Diese Vorhersage tritt nur bei dem Datensatz auf, der alle Datenpunkte beinhaltet. Es ist also wichtig, dass der Input für das neuronale Netz so nah wie möglich an der Realität ist.
